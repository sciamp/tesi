\chapter{Regola di Littlewood-Richardson}

\begin{prop}
Sia
\begin{math}
\begin{bmatrix}
u_1 & \ldots & u_m\\
v_1 & \ldots & v_m
\end{bmatrix}
\end{math}
in ordine lessicografico e corrispondente secondo
Robinson-Schensted-Knuth alla coppia di tableaux $(P,Q)$. Dato un
qualsiasi tableau $T$ inseriamo gli elementi della seconda riga
$T' = ((\ldots (T \gets v_1) \gets \ldots ) \gets v_{m-1} ) \gets v_m$ e
collochiamo successivamente gli $u_1, \ldots, u_m$ in un diagramma
vuoto nelle posizioni occupate dai corrispettivi $v_i$ in $T'$,
otteniamo quindi uno skew tableu $S$.\\
Lo skew tableau $S$ si rettifica a $Rect(Q)=U$.
\end{prop}

Dati tre diagrammi di Young $\lambda \vdash n$, $\mu \vdash m$ e $\nu
\vdash r$ ci chiediamo in quanti modi un tableau $V$ che riempie
$\nu$ pu\`o essere scritto come prodotto di due tableaux $T$ e $U$
che riempiono rispettivamente $\lambda$ e $\mu$.

\begin{oss}
Affinch\'e una tale fattorizzazione esista (ovvero affinch\'e il numero
di modi sia non nullo) \`e necessario che siano verificate le seguenti
condizioni:
\begin{enumerate}[(i)]
\item $r = n + m$
\item $\lambda \subset \nu$
\end{enumerate}
\end{oss}

\begin{defn}
Dati due tableaux $U_0$ e $V_0$ di forma $\mu$ e $\nu$
rispettivamente, definiamo
\begin{center}\begin{math}
\mathcal{S}(\nu/\lambda, U_0) = \{\text{skew tableaux } S \text{ su
}\nu/\lambda \text{ tali che } Rect(S) =U_0\}
\end{math}\\
\begin{math}
\mathcal{T}(\lambda, \mu, V_0)=\{(T,U) \mid T \text{ tableau su }
\lambda, U \text{ tableau su } \mu, T\cdot U=V_0\}.
\end{math}\end{center}
\end{defn}

\begin{prop}
Per ogni tableau $U$ che riempie $\mu$ e ogni tableau $V$ su $\nu$
esiste una corrispondenza biunivoca tra gli insiemi $\mathcal{T}(\lambda, \mu, V)$
e $\mathcal{S}(\nu/\lambda, U)$
\end{prop}

Da cui l'importante

\begin{cor}
$ | \mathcal{S}(\nu/\lambda, U_0) |$ e $| \mathcal{T}(\lambda, \mu,
  V_0) |$ non dipendono da $U_0$ e $V_0$ ma solamante da $\lambda$,
  $\mu$ e $\nu$.
\end{cor} 

\begin{defn}[Numeri di Littlewood-Richardson]
Chiamiamo numeri di Littlewood-Richardson $ c_{\lambda \mu}^{\nu} = |
\mathcal{S}(\nu/\lambda, U_0) | = | \mathcal{T}(\lambda, \mu, V_0) |$.
\end{defn}

\begin{cor}
In $R_{[m]}$ vale $S_{\lambda}[m] \cdot S_{\mu}[m] =
\sum\limits_{\nu} c_{\lambda \mu}^{\nu} S_{\nu}[m]$.
\end{cor}

\begin{oss}
Applicando al corollario precedente l'omomorfismo \eqref{ring_tab}
otteniamo
\begin{center}\begin{math}
s_{\lambda}(x_1,\ldots,x_m) \cdot s_{\mu}(x_1,\ldots,x_m) =
\sum\limits_{\nu} c_{\lambda \mu}^{\nu} s_{\nu}(x_1,\ldots,x_m)
\end{math}\end{center}
che ci permette di esprimere il prodotto di due polinomi di Schur come
combinazione lineare di altri polinomi di Schur dove i coefficienti
sono proprio i numeri di Littlewood-Richardson.
\end{oss}

\section{Un algoritmo per i numeri di Littlewood-Richardson}
Nella definizione dei numeri di Littlewood-Richardson non viene
fornita una procedura di calcolo esplicita, tuttavia \`e possibile
ricavarne una dalla descrizione che segue.

\begin{defn}[Reverse lattice word]
Una parola $w=x_1\ldots x_r$, con $x_i$ interi positivi, si dice una reverse lattice word se per
ogni $1 \leq k \leq r$ si ha che il numerio di lettere presenti in
$\bar w_k = x_r,\ldots,x_k$ uguali a $i$ \`e
al pi\`u uguale al numero di lettere uguali a $i-1$ per ogni $i \geq
1$.
\end{defn}

\begin{defn}[Skew tableau di Littlewood-Richardson]
Uno skew tableau si dice di Littlewood-Richardson se la sua parola
$w(T)$ \`e una reverse lattice word.
\end{defn}

\begin{defn}[Contenuto di un tableau o di uno skew tableau]
Un tableau o uno skew tableau ha contenuto $\mu = (\mu_1, \ldots,
\mu_h)$ se le sue entrate sono costituite da $\mu_1$ volte $1$,
\ldots, $\mu_h$ volte $h$.
\end{defn}

\begin{oss}
Possiamo sempre supporre, a meno di riordinare $\mu_1, \ldots,\mu_h$,
che $\mu$ sia una partizione (e quindi un diagramma di Young).
\end{oss}

\begin{prop}
Il numero di skew tableaux di Littlewood-Richardson che riempiono lo
skew diagram $\nu/\lambda$ con contenuto $\mu$ \`e $c_{\lambda \mu}^{\nu}$.
\end{prop}

Vediamo l'implementazione:
\begin{alltt}
/* We should check that the three diagrams are suitable for Littlewood Richardson */
/* rev_big_shape, rev_small_shape are reversed words of diagrams */
/* gen_skew_tab should be [], and curr should be 1 */
\emph{generic\_skew\_tableau} (rev\_big\_shape, rev\_small\_shape, gen\_skew\_tab, curr) := block (
  [],
  if curr <= length (rev\_small\_shape) then block (
    [empty\_box,zeros,infs,current\_line],
    empty\_box : rev\_small\_shape[curr],
    zeros : makelist (0, empty\_box),
    infs : makelist (inf, rev\_big\_shape[curr] - empty\_box),
    current\_line : [append (zeros, infs)],
    return (\emph{generic\_skew\_tableau} (rev\_big\_shape, rev\_small\_shape, append (current\_line, gen\_skew\_tab), curr + 1)))
  else if curr <= length (rev\_big\_shape) then block (
    [current\_line],
    current\_line : [makelist (inf, rev\_big\_shape[curr])],
    return (\emph{generic\_skew\_tableau} (rev\_big\_shape, rev\_small\_shape, append (current\_line, gen\_skew\_tab), curr + 1)))
  else return (gen\_skew\_tab));

/* Remark: in filling a transposed Littlewood-Richardson skew tableau, beginning from bottom left */
/* we must start with 1 (otherwise the corresponding word is not a lattice word, and hence the corresponding */
/* word of the skew tableau is not a reverse lattice word). For preserving the structure of the tableau */
/* the first column should be filled with ones. */
/* A proof of this remark is in my thesis (http://linkgoeshere). */
/* st should be a transposed generic skew tableau, fst should be [] and level should be 1. */
\emph{fill\_first\_line} (st, u) := block (
  [filled\_skew\_tableau, ones],
  filled\_skew\_tableau : copylist (st),
  ones : lsum (i, i, map (lambda ([x], x[1] : x[1]/inf), filled\_skew\_tableau)),
  /* we should check if there are enough ones in the content u to fill the first column! */
  if ones <= u[1] then return ([filled\_skew\_tableau, ones]) else return ([]));

\emph{reverse\_lattice\_word\_condition} (u, u\_ins, try) :=
if (try > 1) and (u[try] > u\_ins[try]) and (u\_ins[try] + 1 <= u\_ins [try-1]) then true
else if (try = 1) and (u[1] > u\_ins[1]) then true
else false;

/* u\_len should be length (u) */
\emph{young\_tableau\_condition} (curr, u\_len, i, j, try) :=
if ((i>1) and (length(curr[i-1])>=j) and (try<=curr[i-1][j])) or ((i>1) and (length(curr[i-1])<j)) or (i=1) then block (
  if (j>1) and (i<length(curr)) and (curr[i][j-1]<try) and (try<=u\_len-length(curr[i])+j) then true
  else if (j=1) and (i<length(curr)) and (try<u\_len-length(curr[i])+j) then true
  else if (j>1) and (i=length(curr)) and (curr[i][j-1]<try) and (try<=u\_len-length(curr[i])+j) then true
  else false)
else false;

/* try should be 1 */
/* nextl should be [] */
/* curr[i][j] is empty, we should check this before calling \emph{next\_element} */
\emph{next\_element} (curr, u, u\_ins, i, j, try, nextl) :=
if (try <= length (u)) then block (
  [],
  if \emph{reverse\_lattice\_word\_condition} (u, u\_ins, try) and \emph{young\_tableau\_condition} (curr, length (u), i, j, try) then block (
    [nexte, next\_u\_ins],
    nexte : copylist (curr),
    next\_u\_ins : copylist (u\_ins),
    nexte[i][j] : try,
    next\_u\_ins[try] : next\_u\_ins[try] + 1,
    return (\emph{next\_element} (curr, u, u\_ins, i, j, try + 1, append (nextl, [[nexte, next\_u\_ins]]))))
  else return (\emph{next\_element} (curr, u, u\_ins, i, j, try + 1, nextl)))
else return (nextl); 

/* Returns a list containing the list of Littlewood-Richardson skew tableaux */
/* with a given shape d/e and a given content u (i.e. Littlewood-Richardson number) */
/* Actually d,e,u will be reversed words of transposed tableaux */
/* d,e are reversed words of diagram, */
/* u is a reversed words of a diagram with u[1]-(second (\emph{generic\_skew\_tableau} (d, e, [], 1), [])) */
/* u\_ins keeps track of which and how many letters have been used, should be makelist (0, length (u)) */
/* st should be the transposed generic skew tableau with first line filled with ones */
/* i.e. \emph{fill\_first\_line} (remove\_tableau\_column (first (\emph{generic\_skew\_tableau} (d, e, [], 1), []))) */
/* tl is the tableaux list, should be [[st, u\_ins]] */
/* i,j are index to move around the skew tableau like in a matrix, i sould be 1 and j 2 */
/* This function should be called only if \emph{fill\_first\_line} returns non empty list. */
\emph{littlewood\_richardson\_num} (d, e, u, tl, i, j) := block (
  [],
  /* WARNING! do not change the order in the if condition! */
  if (not emptyp (tl)) and (i <= length (tl[1][1]))then block (
    [curr],
    curr : first (tl), /* curr[1] is a skew tableau, curr[2] is the crresponding u\_ins */
    if (j <= length (curr[1][i])) then block (
      [],
      if (curr[1][i][j] # inf) then return (\emph{littlewood\_richardson\_num} (d, e, u, tl, i, j+1))
      else block (
        [next\_l],
        tl : delete (curr, tl),
        next\_l : \emph{next\_element} (curr[1], u, curr[2], i, j, 1, []),
        tl : append (tl, next\_l),
        return (\emph{littlewood\_richardson\_num} (d, e, u, tl, i, j))))
    else return (\emph{littlewood\_richardson\_num} (d, e, u, tl, i+1, 1))) /* end of line, go to next */
  else return (tl));
\end{alltt}


Dati tre diagrammi di Young $\nu$,$\lambda$ e $\mu$ la funzione
\emph{littlewood\_richardson\_num} restituisce una lista contenente
tutti i possibili skew tableau di Littlewood-Richardson che riempiono
$\nu/\lambda$ con contenuto $\mu$. Dal momento che nel riempire lo
skew diagram dobbiamo costruire una reverse lattice word, lavoriamo
con i diagrammi coniugati e riempiamo lo skew diagram
$\tilde{\nu}/\tilde{\lambda}$ (coniugato di $\nu/\lambda$).
Il funzionamento \`e analogo a quello dell'algoritmo per il calcolo
dei polinomi di Schur: si calcolano i possibili riempimenti, coerenti
con i requisiti di reverse lattice word e di Young tableau, di un
elemento e per ognuno di essi si calcolano i possibili riempimenti
dell'elemento successivo, abbandonando di volta in volta eventuali
riempimenti impossibili, fino a quando abbiamo una lista di
riempimenti (completi) dello skew diagram $\tilde{\nu}/\tilde{\lambda}$.

La funzione \emph{fill\_first\_line} che si occupa di riepire la prima
colonna dello skew taleau $\tilde{\nu}/\tilde{\lambda}$ con 1, infatti:

\begin{oss}
La prima riga di uno skew tableau di Littlewood-Richardson ha tutte le
entrate pari a 1.
\begin{proof}
L'ultimo elemento della prima riga deve necessariamente essere pari a
1, altrimenti la parola risultante non terminerebbe per 1 e non
sarebbe quindi una reverse lattice word. Allora poich\'e le entrate
devono rispettare la (i) della definizione \ref{ytab} gli altri
elementi della prima riga, precedendo l'ultimo elemento, devono
necessariamente essere pari a 1.
\end{proof}
\end{oss}
%% da commentare la funzione fill_first_line
%% /* Remark: in filling a transposed Littlewood-Richardson skew tableau, beginning from bottom left */
%% /* we must start with 1 (otherwise the corresponding word is not a lattice word, and hence the corresponding */
%% /* word of the skew tableau is not a reverse lattice word). For preserving the structure of the tableau */
%% /* the first column should be filled with ones. */
%% /* A proof of this remark is in my thesis (http://linkgoeshere). */
%% /* st should be a transposed generic skew tableau, fst should be [] and level should be 1. */
