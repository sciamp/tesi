\chapter{Regola di Littlewood-Richardson}
% Non sono sicuro di doverne parlare. Per ora lascio in sospeso.
\section{Corrispondenza di Robinson-Schensted-Knuth}\label{RSKcorr}
Data una parola $w=x_1\ldots x_r$ indichiamo con $P(w)$ l'\emph{unico} tableau con
parola Knuth-equivalente a $w$ (teorema \ref{knuth_equiv_word}). Un
modo per ottenede $P(w)$ data $w$ \`e la \emph{procedura canonica}
vista nel capitolo precedente. Mentre costruiamo il tableau $P(w)$
teniamo traccia delle posizioni in cui stiamo inserendo le lettere di
$w$ costruendo un'altra tabella $Q(w)$ della stessa forma di $P(w)$
inserendo $k$ in posizione $Q(w)^i_j$ ogniqualvolta inseriamo $x_k$
nella riga $i$-esima alla colonna $j$-esima del tableau $P(w)$. Dunque
se stiamo costruendo $P_k=P(x_1\ldots x_k)=P_{k-1} \gets x_k =
((\ldots (\varnothing \gets x_1) \gets \ldots ) \gets x_{k-1} ) \gets
x_k$ inseriremo $k$ in $Q_k=Q(x_1 \ldots x_k)$ in corrispondenza della posizione
presente in $P_k$ ma non in $P_{k-1}$.

\begin{oss}
Ad ogni passo $1 \leq k \leq r$ della procedura canonica, $Q_k$ \`e un
tableau standard.
%% \begin{proof}
%% Chiaramente le entrate di $Q(w)$ sono tutte distinte fra loro essendo
%% indici delle lettere della parola $w$. Supponiamo che $Q_k$ sia un
%% tableau standard, si hanno tre casi:\\
%% $x_k < x_{k+1}$: \\
%% $x_k = x_{k+1}$:\\
%% $x_k > x_{k+1}$:\\
%% \end{proof}
\end{oss}

Essendo note le posizioni in cui sono state inserite le varie lettere,
da $(P(w),Q(w))$ possiamo ricostruire $w$, questo significa che data
una qualunque coppia di tableaux $(P,Q)$ che riempiono lo stesso
diagramma con $Q$ standard esiste una parola $\bar w$ tale che
$(P,Q)=(P(\bar w),Q(\bar w))$.

Abbiamo cos\`i una corrispondenza
biunivoca fra le parole $w$ di lunghezza $r$ con lettere
nell'alfabeto $[n]$ e le coppie $(P,Q)$ di tableaux riempimenti dello
stesso diagramma $\lambda \vdash r$ con $P$ ad entrate in $[n]$ e $Q$
standard. Nel caso particolare in cui $r=n$ e le lettere di $w$ siano
tutte diverse una dall'altra, ovvero $w \in S_n$, si ha che anche $P$
\`e uno standard tableau e viceversa.

\begin{notaz}[Vettore doppio]
Chiamiamo vettore doppio una matrice $2 \times n$. 
\end{notaz}

\begin{defn}
Un vettore doppio
\begin{math}
\begin{bmatrix}
u_1 & \ldots & u_m\\
v_1 & \ldots & v_m
\end{bmatrix}
\end{math}
si dice in ordine lessicografico se valgono:
\begin{enumerate}[(i)]
\item $u_1 \leq \ldots \leq u_n$;
\item $v_i \leq v_{i+1}$ se $u_1 = u_{i+1}$.
\end{enumerate}
\end{defn}

Knuth generalizz\`o (per i dettagli si veda ~\cite{fulton1997young})
la corrispondenza precedente ad una qualunque coppia di
tableau $(P,Q)$, che riempiono lo stesso diagramma, con $P$ ad entrate
in $[n]$ e $Q$ in $[m]$. In questo caso si ha la corrispondenza fra le
coppie $(P,Q)$ e i \emph{vettori doppi} in ordine lessicografico
\begin{math}
\begin{bmatrix}
u_1 & \ldots & u_m\\
v_1 & \ldots & v_m
\end{bmatrix}
\end{math}.

\begin{prop}
Sia
\begin{math}
\begin{bmatrix}
u_1 & \ldots & u_m\\
v_1 & \ldots & v_m
\end{bmatrix}
\end{math}
in ordine lessicografico e corrispondente secondo
Robinson-Schensted-Knuth alla coppia di tableaux $(P,Q)$. Dato un
qualsiasi tableau $T$ inseriamo gli elementi della seconda riga
$T' = ((\ldots (T \gets v_1) \gets \ldots ) \gets v_{m-1} ) \gets v_m$ e
collochiamo successivamente gli $u_1, \ldots, u_m$ in un diagramma
vuoto nelle posizioni occupate dai corrispettivi $v_i$ in $T'$,
otteniamo quindi uno skew tableu $S$.\\
Allora lo skew tableau $S$ si rettifica a $Rect(S)=Q$.
\end{prop}

\section{I numeri di Littlewood-Richardson}

Dati tre diagrammi di Young $\lambda \vdash n$, $\mu \vdash m$ e $\nu
\vdash r$ ci chiediamo in quanti modi un tableau $V$ che riempie
$\nu$ pu\`o essere scritto come prodotto di due tableaux $T$ e $U$
che riempiono rispettivamente $\lambda$ e $\mu$.

\begin{oss}
Affinch\'e una tale fattorizzazione esista (ovvero affinch\'e il numero
di modi sia non nullo) \`e necessario che siano verificate le seguenti
condizioni:
\begin{enumerate}[(i)]
\item $r = n + m$
\item $\lambda \subset \nu$
\end{enumerate}
\end{oss}

\begin{defn}
Dati due tableaux $U_0$ e $V_0$ di forma $\mu$ e $\nu$
rispettivamente, definiamo
\begin{center}\begin{math}
\mathcal{S}(\nu/\lambda, U_0) = \{\text{skew tableaux } S \text{ su
}\nu/\lambda \text{ tali che } Rect(S) =U_0\}
\end{math}\\
\begin{math}
\mathcal{T}(\lambda, \mu, V_0)=\{(T,U) \mid T \text{ tableau su }
\lambda, U \text{ tableau su } \mu, T\cdot U=V_0\}.
\end{math}\end{center}
\end{defn}

\begin{prop}
Per ogni tableau $U$ che riempie $\mu$ e ogni tableau $V$ su $\nu$
esiste una corrispondenza biunivoca tra gli insiemi $\mathcal{T}(\lambda, \mu, V)$
e $\mathcal{S}(\nu/\lambda, U)$
\end{prop}

Da cui l'importante

\begin{cor}
$ | \mathcal{S}(\nu/\lambda, U_0) |$ e $| \mathcal{T}(\lambda, \mu,
  V_0) |$ non dipendono da $U_0$ e $V_0$ ma solamante da $\lambda$,
  $\mu$ e $\nu$.
\end{cor} 

\begin{defn}[Numeri di Littlewood-Richardson]
Chiamiamo numeri di Littlewood-Richardson $ c_{\lambda \mu}^{\nu} = |
\mathcal{S}(\nu/\lambda, U_0) | = | \mathcal{T}(\lambda, \mu, V_0) |$.
\end{defn}

\begin{cor}
In $R_{[m]}$ vale $S_{\lambda}[m] \cdot S_{\mu}[m] =
\sum\limits_{\nu} c_{\lambda \mu}^{\nu} S_{\nu}[m]$.
\end{cor}

\begin{oss}\label{lr_formula_pol_schur}
Applicando al corollario precedente l'omomorfismo \eqref{ring_tab}
otteniamo
\begin{center}\begin{math}
s_{\lambda}(x_1,\ldots,x_m) \cdot s_{\mu}(x_1,\ldots,x_m) =
\sum\limits_{\nu} c_{\lambda \mu}^{\nu} s_{\nu}(x_1,\ldots,x_m)
\end{math}\end{center}
che ci permette di esprimere il prodotto di due polinomi di Schur come
combinazione lineare di altri polinomi di Schur dove i coefficienti
sono proprio i numeri di Littlewood-Richardson.
\end{oss}

\section{Un algoritmo per i numeri di Littlewood-Richardson}
Nella definizione dei numeri di Littlewood-Richardson non viene
fornita una procedura di calcolo esplicita, tuttavia \`e possibile
ricavarne una dalla descrizione che segue.

\begin{defn}[Reverse lattice word]
Una parola $w=x_1\ldots x_r$, con $x_i$ interi positivi, si dice una reverse lattice word se per
ogni $1 \leq k \leq r$ si ha che il numerio di lettere presenti in
$\bar w_k = x_r,\ldots,x_k$ uguali a $i$ \`e
al pi\`u uguale al numero di lettere uguali a $i-1$ per ogni $i \geq
1$.
\end{defn}

\begin{defn}[Skew tableau di Littlewood-Richardson]
Uno skew tableau si dice di Littlewood-Richardson se la sua parola
$w(T)$ \`e una reverse lattice word.
\end{defn}

\begin{defn}[Contenuto di un tableau o di uno skew tableau]
Un tableau o uno skew tableau ha contenuto $\mu = (\mu_1, \ldots,
\mu_h)$ se le sue entrate sono costituite da $\mu_1$ volte $1$,
\ldots, $\mu_h$ volte $h$.
\end{defn}

\begin{oss}
Possiamo sempre supporre, a meno di riordinare $\mu_1, \ldots,\mu_h$,
che $\mu$ sia una partizione (e quindi un diagramma di Young).
\end{oss}

\begin{prop}
Il numero di skew tableaux di Littlewood-Richardson che riempiono lo
skew diagram $\nu/\lambda$ con contenuto $\mu$ \`e $c_{\lambda \mu}^{\nu}$.
\end{prop}

Dati tre diagrammi di Young $\nu$,$\lambda$ e $\mu$ la funzione
\emph{littlewood\_richardson\_num} restituisce una lista contenente
tutti i possibili skew tableau di Littlewood-Richardson che riempiono
$\nu/\lambda$ con contenuto $\mu$. Dal momento che nel riempire lo
skew diagram dobbiamo costruire una reverse lattice word, lavoriamo
con i diagrammi coniugati e riempiamo lo skew diagram
$\tilde{\nu}/\tilde{\lambda}$ (coniugato di $\nu/\lambda$).
Il funzionamento \`e analogo a quello dell'algoritmo per il calcolo
dei polinomi di Schur: si calcolano i possibili riempimenti, coerenti
con i requisiti di reverse lattice word e di Young tableau, di un
elemento e per ognuno di essi si calcolano i possibili riempimenti
dell'elemento successivo, abbandonando di volta in volta eventuali
riempimenti impossibili, fino a quando abbiamo una lista di
riempimenti (completi) dello skew diagram $\tilde{\nu}/\tilde{\lambda}$.

La funzione \emph{fill\_first\_line} che si occupa di riepire la prima
colonna dello skew taleau $\tilde{\nu}/\tilde{\lambda}$ con 1, infatti:

\begin{oss}
La prima riga di uno skew tableau di Littlewood-Richardson ha tutte le
entrate pari a 1.
\begin{proof}
L'ultimo elemento della prima riga deve necessariamente essere pari a
1, altrimenti la parola risultante non terminerebbe per 1 e non
sarebbe quindi una reverse lattice word. Allora poich\'e le entrate
devono rispettare la (i) della definizione \ref{ytab} gli altri
elementi della prima riga, precedendo l'ultimo elemento, devono
necessariamente essere pari a 1.
\end{proof}
\end{oss}
%% da commentare la funzione fill_first_line
%% /* Remark: in filling a transposed Littlewood-Richardson skew tableau, beginning from bottom left */
%% /* we must start with 1 (otherwise the corresponding word is not a lattice word, and hence the corresponding */
%% /* word of the skew tableau is not a reverse lattice word). For preserving the structure of the tableau */
%% /* the first column should be filled with ones. */
%% /* A proof of this remark is in my thesis (http://linkgoeshere). */
%% /* st should be a transposed generic skew tableau, fst should be [] and level should be 1. */

Vediamo un esempio di utilizzo della funzione che abbiamo realizzato.
Dati due diagrammi \texttt{e:[2,3]} e \texttt{u:[1,2]} calcoliamo la
lista dei possibili diagrammi che sono riempiti dai tableaux che si
ottengono come prodotto di tutti i possibili tableaux su \texttt{e} e
\texttt{u} attaccando al primo un numero di box pari alla dimensione
di \texttt{u} in tutte le posizioni possibili:
\begin{alltt}
ds:unique (\emph{yshape\_product} ([[e,3]])).
\end{alltt}
Calcoliamo quindi tutti i possibili skew diagram che si ottengono da ogni
diagramma della lista \texttt{ds} e dal diagramma \texttt{e},
tasponiamoli, e riempiamone la prima colonna:
\begin{alltt}
skews:map (lambda ([x], \emph{fill\_first\_line} (
  \emph{remove\_tableau\_column} (\emph{generic\_skew\_tableau} (x[1], e, [], 1),[]), u)), ds)
\end{alltt}
da cui otteniamo
\begin{alltt}
\([[[[0,\infty],[0,0],[0,0,\infty,\infty,\infty]],0],[[[0,\infty],[0,0,\infty],[0,0,\infty,\infty]],0],[[[0,\infty,\infty],[0,0,\infty],[0,0,\infty]],0],\)
\([[[0,\infty],[0,0],[0,0,\infty,\infty]],0],[[[0,\infty],[0,0,\infty],[0,0,\infty]],0],[[[1],[0,\infty],[0,0],[0,0,\infty]],1],\)
\([[[1,\infty],[0,\infty],[0,0],[0,0]],1],[[[1],[1],[0,\infty],[0,0],[0,0]],2]]\)
\end{alltt}

Calcoliamo infine i numeri di Littlewood-Richardson corrispondenti ai
diagrammi \texttt{ds}:
\begin{alltt}
l1:map (
    lambda ([x], if (not emptyp (skews[x])) then 
      [ds[x][1],length(\emph{littlewood\_richardson\_num}(d,e,u,
                      [[skews[x][1],append([skews[x][2]],makelist(0,length(u)-1))]],
                      1,2))]),
    makelist (i,i,1,length (skews)))
l2 : delete (false, map (lambda ([x], if (listp (x)) and (x[2] > 0) then x ), l1));
\end{alltt}
da cui si ottiene
\texttt{[[[3,3,1,1],1],[[3,3,2],1],[[4,3,1],2],[[4,4],1],[[5,3],1]]}.
Gli elementi di questa lista sono i diagrammi che costituiscono la combinazione lineare a sinistra
dell'uguale nell'osservazione \ref{lr_formula_pol_schur} accompagnati
dai corrispondenti numeri di Littlewood-Richardson che compaiono nella
stessa formula come coefficienti.\\
Calcoliamo i corrispondenti polinomi di Schur (i.e. applichiamo
l'omomorfismo \eqref{ring_tab}) e quindi la combinazione lineare:
\begin{alltt}
poly:map (lambda ([x], x[2]*\emph{fl\_better\_yschur\_polynomial\_rows} (x[1],5)), l2);
prod\_lr:ratsimp (sum (poly[i],i,1,length(poly)))
\end{alltt}
che possiamo confrontare con il prodotto dei polinomi di Schur
associati ai diagrammi \texttt{e} ed \texttt{u}:
\begin{alltt}
poly\_e:\emph{fl\_better\_yschur\_polynomial\_rows} (e,5)
poly\_u:\emph{fl\_better\_yschur\_polynomial\_rows} (u,5)
prod:ratsimp (poly\_e*poly\_u)
is (expand (prod - prod\_lr = 0))
\(\rightarrow\) true
\end{alltt}
che ci permette di confermare su questo esempio quanto visto nella teoria.
