\chapter{Regola di Littlewood-Richardson}

\begin{prop}
Sia
\begin{math}
\begin{bmatrix}
u_1 & \ldots & u_m\\
v_1 & \ldots & v_m
\end{bmatrix}
\end{math}
in ordine lessicografico e corrispondente secondo
Robinson-Schensted-Knuth alla coppia di tableaux $(P,Q)$. Dato un
qualsiasi tableau $T$ inseriamo gli elementi della seconda riga
$T' = ((\ldots (T \gets v_1) \gets \ldots ) \gets v_{m-1} ) \gets v_m$ e
collochiamo successivamente gli $u_1, \ldots, u_m$ in un diagramma
vuoto nelle posizioni occupate dai corrispettivi $v_i$ in $T'$,
otteniamo quindi uno skew tableu $S$.\\
Lo skew tableau $S$ si rettifica a $Rect(S)=U$.
\end{prop}

Dati tre diagrammi di Young $\lambda \vdash n$, $\mu \vdash m$ e $\nu
\vdash r$ ci chiediamo in quanti modi un tableau $V$ che riempie
$\nu$ pu\`o essere scritto come prodotto di due tableaux $T$ e $U$
che riempiono rispettivamente $\lambda$ e $\mu$.

\begin{oss}
Affinch\'e una tale fattorizzazione esista (ovvero affinch\'e il numero
di modi sia non nullo) \`e necessario che siano verificate le seguenti
condizioni:
\begin{enumerate}[(i)]
\item $r = n + m$
\item $\lambda \subset \nu$
\end{enumerate}
\end{oss}

\begin{defn}
Dati due tableaux $U_0$ e $V_0$ di forma $\mu$ e $\nu$
rispettivamente, definiamo
\begin{center}\begin{math}
\mathcal{S}(\nu/\lambda, U_0) = \{\text{skew tableaux } S \text{ su
}\nu/\lambda \text{ tali che } Rect(S) =U_0\}
\end{math}\\
\begin{math}
\mathcal{T}(\lambda, \mu, V_0)=\{(T,U) \mid T \text{ tableau su }
\lambda, U \text{ tableau su } \mu, T\cdot U=V_0\}.
\end{math}\end{center}
\end{defn}

\begin{prop}
Per ogni tableau $U$ che riempie $\mu$ e ogni tableau $V$ su $\nu$
esiste una corrispondenza biunivoca tra gli insiemi $\mathcal{T}(\lambda, \mu, V)$
e $\mathcal{S}(\nu/\lambda, U)$
\end{prop}

Da cui l'importante

\begin{cor}
$ | \mathcal{S}(\nu/\lambda, U_0) |$ e $| \mathcal{T}(\lambda, \mu,
  V_0) |$ non dipendono da $U_0$ e $V_0$ ma solamante da $\lambda$,
  $\mu$ e $\nu$.
\end{cor} 

\begin{defn}[Numeri di Littlewood-Richardson]
Chiamiamo numeri di Littlewood-Richardson $ c_{\lambda \mu}^{\nu} = |
\mathcal{S}(\nu/\lambda, U_0) | = | \mathcal{T}(\lambda, \mu, V_0) |$.
\end{defn}

\begin{cor}
In $R_{[m]}$ vale $S_{\lambda}[m] \cdot S_{\mu}[m] =
\sum\limits_{\nu} c_{\lambda \mu}^{\nu} S_{\nu}[m]$.
\end{cor}

\begin{oss}
Applicando al corollario precedente l'omomorfismo \eqref{ring_tab}
otteniamo
\begin{center}\begin{math}
s_{\lambda}(x_1,\ldots,x_m) \cdot s_{\mu}(x_1,\ldots,x_m) =
\sum\limits_{\nu} c_{\lambda \mu}^{\nu} s_{\nu}(x_1,\ldots,x_m)
\end{math}\end{center}
che ci permette di esprimere il prodotto di due polinomi di Schur come
combinazione lineare di altri polinomi di Schur dove i coefficienti
sono proprio i numeri di Littlewood-Richardson.
\end{oss}

\section{Un algoritmo per i numeri di Littlewood-Richardson}
Nella definizione dei numeri di Littlewood-Richardson non viene
fornita una procedura di calcolo esplicita, tuttavia \`e possibile
ricavarne una dalla descrizione che segue.

\begin{defn}[Reverse lattice word]
Una parola $w=x_1\ldots x_r$, con $x_i$ interi positivi, si dice una reverse lattice word se per
ogni $1 \leq k \leq r$ si ha che il numerio di lettere presenti in
$\bar w_k = x_r,\ldots,x_k$ uguali a $i$ \`e
al pi\`u uguale al numero di lettere uguali a $i-1$ per ogni $i \geq
1$.
\end{defn}

\begin{defn}[Skew tableau di Littlewood-Richardson]
Uno skew tableau si dice di Littlewood-Richardson se la sua parola
$w(T)$ \`e una reverse lattice word.
\end{defn}

\begin{defn}[Contenuto di un tableau o di uno skew tableau]
Un tableau o uno skew tableau ha contenuto $\mu = (\mu_1, \ldots,
\mu_h)$ se le sue entrate sono costituite da $\mu_1$ volte $1$,
\ldots, $\mu_h$ volte $h$.
\end{defn}

\begin{oss}
Possiamo sempre supporre, a meno di riordinare $\mu_1, \ldots,\mu_h$,
che $\mu$ sia una partizione (e quindi un diagramma di Young).
\end{oss}

\begin{prop}
Il numero di skew tableaux di Littlewood-Richardson che riempiono lo
skew diagram $\nu/\lambda$ con contenuto $\mu$ \`e $c_{\lambda \mu}^{\nu}$.
\end{prop}
