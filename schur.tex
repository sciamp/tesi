\chapter{Polinomi di Schur}

\begin{defn}[Polinomio di Schur]
Data una partizione $\lambda \vdash n$ (i.e. un diagramma di Young
$\lambda = (n_1, \ldots, n_m)$) in $m$ parti (e quindi il diagramma ha
$m$ righe) e un qualunque tableau $T$ che riempie $\lambda$ indichiamo
con $x^T = \prod\limits_{i = 1}^m x^{c(i)}$. Sommando al variare di
$T$ fra i possibili riempimenti di $\lambda$ otteniamo
\begin{math}
s_\lambda(x_1,\ldots,x_m) = \sum x^T
\end{math}
che chiameremo polinomio di Schur.
\end{defn}
