\chapter{Polinomi di Schur}

\begin{defn}[Polinomio di Schur]
Data una partizione $\lambda \vdash n$ (i.e. un diagramma di Young
$\lambda = (n_1, \ldots, n_m)$) in $m$ parti (e quindi il diagramma ha
$m$ righe) e un qualunque tableau $T$ che riempie $\lambda$ indichiamo
con $x^T = \prod\limits_{i = 1}^m x^{c(i)}$. Sommando al variare di
$T$ fra i possibili riempimenti di $\lambda$ otteniamo
\begin{math}
s_\lambda(x_1,\ldots,x_m) = \sum x^T
\end{math}
che chiameremo polinomio di Schur.
\end{defn}

\section{Anello dei tableau}

HO LASCIATO GLI APPUNTI A CASA, CRIBBIO!

\section{Implementazione dei polinomi di Schur}

\begin{alltt}
\emph{yschur\_monomial\_word\_recursive} (flat\_w, mon, i) := block (
  [x],
  if (i>0) then return (\emph{yschur\_monomial\_word\_recursive} (flat\_w,
                                                        mon*x[flat\_w[i]],
                                                        i-1))
  else return (mon));

\emph{last\_or\_zero} (l) := if emptyp (l) then 0 else last(l);

\emph{fill\_next\_line} (len, m, prev, ls) := block (
  [],
  if (len = 0) then return (ls)
  else block (
    [p],
    p : [],
    for x in ls do p : append (p,
                               makelist (append (x, [y]),
                               y,
                               max (\emph{last\_or\_zero} (x) + 1,
                                    prev[length (x)+1]),
                               m - len + 1)),
    return (\emph{fill\_next\_line} (len - 1, m, prev, p))));


\emph{fl\_fill\_line} (len, m, ls) :=
if (len=0) then ls
else \emph{fl\_fill\_line} (len-1, m, xreduce ('append, 
                                       makelist (makelist (append (x, [y]),
                                                 y,
                                                 \emph{last\_or\_zero} (x) + 1,
                                                 m - len +1), x,
                                                 ls)));

\emph{fl\_list\_transposed\_ytableaux} (shape, level, m, ts) :=
if level < length (shape) then
  if level > 0 then
    if not emptyp (flatten (ts)) then block (
      [nls,ts\_first],
      ts\_first : first (ts),
      ts : delete (ts\_first, ts),
      nls : \emph{fill\_next\_line} (shape [level+1], m,
                            rest (ts\_first, -length(ts\_first)+shape[level]),
                            [[]]),
      ts : append (ts, map (lambda ([x], append (x, ts\_first)), nls)),
      if every (lambda ([x],
                        is (length (flatten (x)) = length (flatten
                        (first (ts))))),
                ts) then level : (level + 1),
      \emph{fl\_list\_transposed\_ytableaux} (shape, level, m, ts))
    else []
  else \emph{fl\_list\_transposed\_ytableaux} (shape, level + 1, m,
                                     \emph{fl\_fill\_line} (shape[level + 1], m, ts))
else ts;

\emph{fl\_better\_yschur\_polynomial} (d, m) := block (
  [ts, pol, shape],
  shape : reverse ((\emph{ydiagram\_transpose} (d))@rows),
  print (shape),
  ts : \emph{fl\_list\_transposed\_ytableaux} (shape, 0, m, [[]]),
  pol : 0,
  for i : 1 thru length (ts) do
    pol : pol + \emph{yschur\_monomial\_word\_recursive} (ts[i], 1, length (ts[i])),
  return (pol));
\end{alltt}

Il cuore dell'algoritmo risiede nella funzione
\emph{fl\_list\_transposed\_ytableaux} che calcola tutti i
possibili tableaux su un dato diagramma $\lambda$. Dal momento che la
definizione \eqref{ytab} pone una condizione
lungo le colonne pi\`u stringente della non decrescenza sulle
righe, calcoliamo i possibili riempimenti del diagramma coniugato
$\tilde{\lambda}$, cos\`i facendo gi\`a dalla prima riga (ovvero la
prima colonna di $\lambda$) riduciamo il numero di riempimenti da
provare.\\
Il riempimento della prima riga avviene da sinistra verso destra
attingendo da $[m]=\{1,\ldots, m\}$ in modo tale che se inseriamo $k \in
[m]$ in posizione $i$-esima, e la prima riga \`e lunga $l_1$, esistano
$k_1,\ldots,k_{l_1-1} \in [m]$ tutti maggiori di $k$. Generiamo quindi
una lista di possibili riempimenti della prima riga, ognuno dei quali
dar\`a luogo ad una lista di possibili riempimenti delle prime due
righe, e cos\`i via.\\ 
Naturalmente riempiendo la riga $j$-esima, alla
posizione $i$ dovremo scegliere $k \in [m]$ tale che esistano
$k_1,\ldots,k_{l_j-1} \in [m]$ tutti maggiori di $k$ e tutti non
minori dell'elemento in posizione $i$-esima della riga
$(j-1)$-esima.\\
Ottenuta infine la lista dei possibili tableaux su $\lambda$ la
funzione \emph{yschur\_monomial\_word\_recursive} calcola il
corrispondente monomio, la somma di questi monomi \`e il polinomio di
Schur cercato $s_\lambda(x_1,\ldots,x_m)$.
