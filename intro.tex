\chapter*{Introduzione}
L'obiettivo principale di questo lavoro \`e lo studio degli
Young tableaux e delle loro applicazioni all'algebra e al calcolo
simbolico. In particolare ci concentreremo sullo sviluppo di una libreria
per le funzioni simmetriche nel sistema di calcolo simbolico Maxima. I
motivi che ci hanno spinto verso la
realizzazione di questa libreria sono molteplici. Primo fra tutti
l'idea di sperimentare con esempi le affermazioni teoriche
contenute in teoremi e proposizioni. Inoltre intendevamo realizzare
una implementazione allo stesso tempo leggibile ed eseguibile degli
algoritmi da affiancare alle descrizioni che si trovano in
letteratura.

Per tutti gli aspetti teorici faremo sempre riferimento, se non diversamente
specificato, al libro di Fulton~\cite{fulton1997young}. Per i
paragrafi \ref{rowbump_par} e \ref{RSKcorr} \`e possibile trovare
approfondimenti sul testo di Knuth~\cite{knuthart}.
Per quanto riguarda l'uso, la sintassi e le funzioni di
libreria di Maxima si fa riferimento alla pagina del
progetto~\cite{maxima_doc}.

Per motivi di tempo e spazio non rientrano negli obiettivi di questo
lavoro l'analisi della complessit\`a
degli algoritmi e lo studio delle possibili ottimizzazioni a livello
implementativo.

Nel primo capitolo forniremo le definizioni di diagramma e tableau di
Young ed introdurremo l'operazione di prodotto fra tableaux. Nel corso del secondo
capitolo introdurremo i polinomi di Schur. Infine nel terzo capitolo
descriveremo i numeri di Littlewood-Richardson ed una loro
applicazione al prodotto dei polinomi di Schur.

Il codice scritto durante il lavoro di tesi \`e rilasciato con licenza libera ed \`e
reperibile all'indirizzo
\url{https://github.com/sciamp/Young_Tableaux/}.
Allo stato attuale la libreria \`e funzionante e liberamente utilizzabile,
restano delle questioni di carattere tecnico da affrontare,
che la rendono ancora poco intuitiva per la diffusione ad un pubblico pi\`u ampio.
Tali questioni non sono state affrontate per motivi di tempo, ma
potrebbero costituire
un importante spunto per un lavoro futuro. In ogni caso tutte le procedure implementate sono
state testate su numerosi esempi espliciti per verificarne anche empiricamente
la loro correttezza.
