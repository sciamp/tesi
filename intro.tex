\chapter*{Introduzione}
L'obiettivo del lavoro di tesi svolto consiste nello studio degli
Young tableaux e nella realizzazione di una libreria per il Computer
Algebra System Maxima. I motivi che hanno spinto verso la
realizzazione di questa libreria sono molteplici. Primo fra tutti il realizzare una sorta
di laboratorio dove sperimentare con esempi le affermazioni teoriche
contenute in teoremi e proposizioni. Inoltre intendevamo realizzare
una implementazione allo stesso tempo leggibile ed eseguibile degli
algoritmi da affiancare alle descrizioni che si trovano in
letteratura.\\
Fatta eccezione per i paragrafi che descrivono l'implementazione che
abbiamo realizzato si fa sempre riferimento, se non diversamente
specificato, al libro di Fulton ~\cite{fulton1997young}. Inoltre per i
paragrafi \ref{rowbump_par} e \ref{RSKcorr} \`e possibile
approfondire sul testo di Knuth ~\cite{knuthart}.\\
Per quanto riguarda il funzionamento, la sintassi e le funzioni di
libreria di Maxima si fa riferimento alla pagina del progetto ~\cite{maxima_doc}.\\
Per motivi di tempo e spazio non rientrano negli obiettivi di questo
lavoro l'analisi della complessit\`a
degli algoritmi e lo studio delle possibili ottimizzazioni a livello
implementativo.

Nel primo capitolo forniremo le definizioni di diagramma e tableau di
Young ed introdurremo un'operazione di prodotto. Nel corso del secondo
capitolo introdurremo i polinomi di Schur. Infine nel terzo capitolo
descriveremo i numeri di Littlewood-Richardson ed una loro
applicazione al prodotto dei polinomi di Schur.

Il codice scritto durante il lavoro di tesi \`e rilasciato con licenza libera ed \`e
completamente reperibile all'indirizzo
\url{https://github.com/sciamp/Young_Tableaux/}.\\
Allo stato attuale la libreria \`e funzionante e completamente utilizzabile,
ovviamente ci sono questioni di carattere tecnico di cui preoccuparsi,
che la rendono lontana dall'essere pronta per il ``grande pubblico'',
che per motivi di tempo non sono state affrontate ma che costituiscono
un importante spunto per un lavoro futuro.
